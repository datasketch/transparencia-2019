\documentclass[letterpaper]{article} 

% \documentclass[]{article}



\usepackage{geometry}
\geometry{letterpaper, top=3.5cm, left=2cm, right=3.5cm, bottom=1.0cm}

\usepackage[compact]{titlesec}
\usepackage[document]{ragged2e}
\usepackage{multirow}
\usepackage{colortbl}

    \usepackage{lmodern}
    \usepackage{amssymb,amsmath}
\usepackage{ifxetex,ifluatex}
\usepackage{fixltx2e} % provides \textsubscript
\ifnum 0\ifxetex 1\fi\ifluatex 1\fi=0 % if pdftex
\usepackage[T1]{fontenc}
\usepackage[utf8]{inputenc}

\usepackage{color}


% pandoc syntax highlighting
  
    
% graphix
\usepackage{graphicx}
\setkeys{Gin}{width=\linewidth,totalheight=\textheight,keepaspectratio}

% booktabs
\usepackage{booktabs}

% float images
\usepackage{wrapfig}

\usepackage{array}


\usepackage[scaled]{helvet}
\renewcommand\familydefault{\sfdefault} 
\usepackage[T1]{fontenc}


\renewcommand{\arraystretch}{0}
\usepackage{ejbi}

\usepackage{fancyhdr}

\pagestyle{fancy}
\fancyhf{}
\fancyhead[LE,RO]{\includegraphics[width=50mm]{logo.png}}

% \pagestyle{fancy}
% \fancyhf{}
% \fancyhead[R]{\includegraphics[width=50mm]{logo.png}}
% \fancyfoot[R]{hola}
% % 
% % \setlength{\headheight}{10.50mm}
% \pagestyle{fancy}

\begin{document}



\begin{table}[h!]
 \setlength{\tabcolsep}{0pt}
\def\arraystretch{0}
  \begin{tabular}{m{4cm}m{12cm}}

    &
    \vspace{10pt} 

                \begin{center}
        \begin{ejbi-title}
        Se pasaron de la raya
        \end{ejbi-title}
        \end{center}
        %\bigskip
                      \begin{center}
      \begin{ejbi-subtitle}
      Sancionados exdiputados de Santander por otorgar facultades para
      modificar la estructura interna
      \end{ejbi-subtitle}
      \end{center}
      %}
      %\bigskip
        \end{tabular}
%  \egroup
\end{table}


%\begin{abstract}
\color{colT}{\justify{En 2013, los diputados del departamento de Santander emitieron una
ordenanza con el fin de otorgar facultades al contralor para la
modificación de la estructura organizacional de la entidad, el Manual de
Funciones y Competencias, el Manual de Operaciones y Procedimientos, las
cargas laborales, la escala salarial de la Contraloría General de
Santander y otras decisiones como suprimir el cargo de conductor y crear
uno de profesional universitario. La Procuraduría General de la Nación
decidió abrir investigación contra 12 diputados por estos hechos, pues
la ordenanza era inconstitucional y habrían incurrido en extralimitación
de sus funciones. El gobernador encargado de la época y el jefe de la
oficina de asesoría jurídica de la Gobernación también fueron
investigados por haber sancionado dicha ordenanza. En abril de 2016, los
diputados fueron citados a audiencia pública y en diciembre de 2017
fueron sancionados por 9 meses al considerarse esta actuación como una
falta grave cometida con culpa gravísima, teniendo en cuenta el grave
daño social de la conducta y la afectación de derechos fundamentales.}}
%\end{abstract}


\vspace{0.5cm}
  

  
\vspace{0.5cm}

\begin{minipage}[t]{0.45\textwidth}%
  
\begin{tabular}{m{3.4cm}m{3.6cm}}
 \begin{ejbi-colone}LUGAR DEL HECHO:\end{ejbi-colone}& 
  \begin{ejbi-coltwo} SANTANDER \end{ejbi-coltwo}\\ 
 \colrul 
 \addlinespace
 \begin{ejbi-colone}FECHA DEL HECHO:\end{ejbi-colone} &
  \begin{ejbi-coltwo} 2013 \end{ejbi-coltwo}   \\
 \colrul 
 \addlinespace
 \specialcell[]{\begin{ejbi-colone}ACTOR O ENTIDAD\end{ejbi-colone} \\
 \addlinespace
 \begin{ejbi-colone}INVOLUCRADO: \end{ejbi-colone}}&
  \begin{ejbi-coltwo} Luis Eduardo Díaz Mateus (Diputado (2012015)). Álvaro Celis Carrillo
(Diputado (2012015)). Camilo Andrés Arenas Valdivieso (Diputado
(2012015)). Édgar Higinio Villabona Carrero (Diputado (2012015)). Fernán
Gabriel Rueda Domínguez (Diputado (2012015)). Henry Hernández Hernández
(Diputado (2012015)). Iván Fernando Aguilar Zambrano (Diputado
(2012015)). Jorge Eliecer García Jaimes (Diputado (2012015)). José Ángel
Ibáñez Almeida (Diputado (2012015)). Luis Fernando Peña Riaño (Diputado
(2012015)). Luis Tulio Tamayo Tamayo (Diputado (2008011)). Rubiela
Vargas González (Diputada (2012015)) \end{ejbi-coltwo} \\ 
\addlinespace\colrul
 \addlinespace
 \specialcell[]{\begin{ejbi-colone}TIPO DE \end{ejbi-colone}\\ 
 \begin{ejbi-colone}CORRUPCIÓN:\end{ejbi-colone}}&
  \begin{ejbi-coltwo} Corrupción Administrativa \end{ejbi-coltwo} \\
 \addlinespace \colrul
 \end{tabular}
\end{minipage}%
\qquad{\color{colfich}\vrule}\qquad
\begin{tabular}{@{}l@{}}

\begin{tabular}{m{3.6cm}m{3.6cm}}
\begin{ejbi-colone}DERECHO VULNERADO\end{ejbi-colone} &
 \begin{ejbi-coltwo}Derechos sociales, económicos y culturales\end{ejbi-coltwo}\\ 
\colrul
\addlinespace
\begin{ejbi-colone}SECTOR AFECTADO\end{ejbi-colone} &
 \begin{ejbi-colone}FUNCIÓN PÚBLICA\end{ejbi-colone}\\ 
\colrul
\addlinespace
\multicolumn{1}{c !{\color{colfich}\vline}}{{\begin{tabular}{@{}c@{}}
 \\
 \\
 \begin{ejbi-colone}ENTIDAD DE\end{ejbi-colone}\\\addlinespace \begin{ejbi-colone}CONOCIMIENTO:\end{ejbi-colone}\end{tabular} }} & 
 \multicolumn{1}{c}{\begin{ejbi-colone}ESTADO JUDICIAL:\end{ejbi-colone}} 
 \\
 \multicolumn{1}{c !{\color{colfich}\vline}}{{\begin{ejbi-coltwo} \parbox[t]{3.4cm}{\begin{center}Asamblea Departamental de Santander\end{center}}\end{ejbi-coltwo}}}  & \multicolumn{1}{c}{\begin{ejbi-coltwo}\parbox[t]{3.4cm}{\begin{center} Sancionado disciplinariamente\end{center}} \end{ejbi-coltwo}} \\ \arrayrulecolor{colfich}\hline
\end{tabular}
\end{tabular}

\vspace{1cm}

\begin{flushright}
\textit{Última actualización } 
\end{flushright}

\end{document}